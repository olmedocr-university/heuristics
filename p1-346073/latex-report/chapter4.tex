\chapter{Conclusions}
\label{chapter: conclusions}

\section{Difficulties with the assignment}

\paragraph{}
With the first part, there were not real problems, we could learn the basic linear programming concepts without leaving a, sort of, known environment as the spreadsheet software is. However, the fact that once you obtain a solution you cannot check if it is the optimal one (you can check its correctness but the grade of the assignment depends on the model and the computation of the optimal solution) gave us some mixed feelings as we discovered a couple of times that our solution was not the optimal and therefore correct one.

When we started with MathProg's section, the situation changed, because the form seemed familiar (it looks quite like a regular programming language, you execute it and get syntax errors, etc) but the concept was not even close to what we knew. The introduction of sets, parameters and variables in the way that GLPK/MathProg handles them was a new thing to us, so we required some time to adapt our minds to the way they work. Once you get your brain to think in this way, the real problem remains in the modelling part, as translating the LP task from a mathematical standpoint to MathProg's syntax is trivial (once you already completed the assignment, of course).
Once again, the syntax errors obtained in the output of GLPK's execution were unmeaningful most of the times and hard to fix, it might be due to the fact that we are used to program in more common languages with a huge development behind and a maybe "long" trajectory.

\paragraph{}
As a final thought, this report took longer than expected. It was our first time working with LaTeX and we found it pretty useful and surprisingly helpful with certain tasks (as creating formulas, indexes, tables of content, formulae referencing ...) but pretty tedious to write in some cases (creating tables and spacing them seems like a nightmare to me).

\newpage\null\thispagestyle{empty}\newpage